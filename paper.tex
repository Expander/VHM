\documentclass[11pt,DIV16,twocolumn]{scrartcl}

\usepackage[utf8x]{inputenc}
\usepackage[T1]{fontenc}
\usepackage[ngerman]{babel}
\usepackage{lmodern}
\usepackage{amsmath}
\usepackage{tabularx}
\usepackage{tikz}
\usepackage[format=plain,labelfont={bf}]{caption}
\usepackage[german]{authblk}
\usepackage{hyperref}

\hypersetup{
  pdftitle   = {Das Vier-Hälse-Modell und seine praktische Anwendung},
  pdfauthor  = {Dr. Alexander Voigt, Dr. Bernadette Schorn},
  colorlinks = true,
  linkcolor  = black,
  citecolor  = black,
  urlcolor   = black,
  linkcolor  = black
}

\title{Das Vier-Hälse-Modell und seine praktische Anwendung}
\renewcommand\Authand{ und }
\author{Dr.~Bernadette Schorn}
\author{Dr.~Alexander Voigt}
\affil{Institut f\"{u}r Angewandte Philologie, Rheinische Universit\"{a}t Flensburg}
\date{11.~November 2021}

\begin{document}
\maketitle

\section*{Kurzdarstellung}

In der vorliegenden Forschungsarbeit wird eine Adaption des
Vier-Seiten-Modells auf den in Westdeutschland vorkommenden
rheinischen Dialekt vorgestellt.  Die Anwendbarkeit dieses neuen
sogenannten Vier-Hälse-Modells wird im Rahmen einer
Vergleichsuntersuchung in Alltagssituationen in Bonn-Mehlem sowie in
Flensburg-Mürwik untersucht.

\section{Das Vier-Hälse-Modell}

Das Vier-Seiten-Modell (VSM) \cite{VSM} ist ein Modell aus dem Bereich
der Kommunikationspsychologie zur Beschreibung verschiedener Ebenen
einer Nachricht, die von einem Sender an einen Empfänger übermittelt
wird.  Gemäß dem VSM enthält jede Nachricht die folgenden vier Ebenen:
\textit{Sachinhalt}, \textit{Selbstoffenbarung},
\textit{Beziehungsaspekt} und \textit{Appell}.
%
Der Sachinhalt beschreibt die Sache auf einer formalen Ebene.  Die
Selbstoffenbarung bezeichnet die Information, die über den Sender der
Nachricht deutlich wird.  Der Beziehungsaspekt enthält Information
darüber, wie der Sender mit dem Empfänger der Nachricht in Beziehung
steht.  Der Appell beschreibt die Absicht, die der Sender mit der
Nachricht erreichen möchte.

Durch die Unterteilung einer Nachricht in die verschiedenen Ebenen
gemäß dem VSM kann die Ursache von Missverständnissen bei der
Kommunikation deutlich werden.  In verschiedenen Regionen Deutschlands
existieren hierfür dedizierte Redewendungen.  Im rheinischen
Sprachraum wird beispielsweise davon gesprochen, "`etwas in den
falschen Hals zu bekommen"'.  Derlei regionale Besonderheiten machen
daher eine Modifikation des VSM erforderlich, in der die lokalen
Charakteristika in der Kommunikation adäquat berücksichtigt werden.
Das von uns postulierte Modell wird in Anlehnung an die o.g.\ Phrase
als Vier-Hälse-Modell (VHM) bezeichnet.  Das VHM unterteilt eine
Nachricht in die folgenden vier Ebenen:\footnote{Diese Unterteilung
  geht auf die im rheinischen Sprachraum verbreiteten folgenden vier
  Sprachen zurück: Hochdeutsch, Durch-die-Nase, Rheinisch und
  Über-andere-Leute.}
%
\begin{itemize}
\item \textit{Sachinhalt}: beschreibt die Sache auf einer formalen
  Ebene.
\item \textit{Naseninhalt} (auch: \textit{Durch-die-Nase sprechen}):
  bezeichnet Laute, die eine emotionale Haltung zu dem Sachverhalt
  deutlich machen.
\item \textit{Dialektinhalt}: beschreibt die Verwendung von
  dialektspezifischen Worten und Redewendungen, i.d.R.\ zur
  Verstärkung oder Abschwächung der übermittelten Aussage.
  % Beispiel: "`Pingelspisser"', "`Don ens
  % de Opnemmer holle!"' oder "`Mer kann et met de Putze och
  % üvverdrieve."'
\item \textit{Fremdinhalt} (auch: \textit{Über-andere-Leute
    sprechen}): bezeichnet den unterschwelligen Teil der Nachricht,
  der eine Aussage über andere Leute macht oder implizite
  Aufforderungen an andere Personen enthält.
  % Beispiel: "`Dat is enne Dreck he!"', "`Do
  % bis enne Schmuddelspitter!"', %, "`De soll ens putze jonn!"' oder
  % "`Don ens sauge-wische-sauge!"'
\end{itemize}
%
Als Beispiel betrachten wir die folgende, häufig vorkommende, Aussage:
"`Igitt, luure ens, de janze Flocke ob de Äd!"'.
%
\begin{center}
  \begin{tabularx}{\linewidth}{lX}
    Sachinhalt: & Auf dem Boden befinden sich Schmutzflocken. \\
    Naseninhalt: & "`igitt"', als Ausdruck der Missbilligung. \\
    Dialektinhalt: & luure ens = schau mal; de janze = die ganzen; Äd = Erde \\
    Fremdinhalt: & Du musst jetzt saugen-wischen-saugen!
  \end{tabularx}
\end{center}
%
Als weiteres Beispiel betrachten wir die Aussage: "`Uh, do is en
Patina ob de Kumod."'
%
\begin{center}
  \begin{tabularx}{\linewidth}{lX}
    Sachinhalt: & Auf der Kommode befindet sich eine Staubschicht. \\
    Naseninhalt: & "`Uh"', als Ausdruck der Missbilligung. \\
    Dialektinhalt: & Patina = Staubschicht; Kumod = Kommode \\
    Fremdinhalt: & Du musst mal Staub wischen!
  \end{tabularx}
\end{center}

\section{Praktische Anwendung}

Im Rahmen einer mehrjährigen Studie wurde die praktische
Anwendbarkeit, sowie die phänomenologischen Implikationen des VHM
untersucht.  Die Untersuchung wurde in Bonn-Mehlem und in
Flensburg-Mürwik mit einer Stichprobengröße von $N=4$ durchgeführt
(Testgruppe).  Als Untersuchungsdesign wurde eine
Vergleichsuntersuchung mit $N=4$ Teilnehmern aus Dresden
(Kontrollgruppe) gewählt (vgl.\ Vier-Bämmen-Modell \cite{VBM}).  Die
Ergebnisse sind in \figurename~\ref{fig:VHM} dargestellt.  Es ist zu
erkennen, dass in der Testgruppe (rheinische Sprachfamilie) der
Dialektanteil statistisch signifikant größer ist, als in der
Kontrollgruppe (sächsische Sprachfamilie).  Dies bedeutet eine
deutlich erhöhte unterschwellige Verstärkung von Aussagen im
rheinischen Sprachraum, die zwangsläufig zu einer durchschnittlich
deutlich erhöhten Putztätigkeit führt, vgl.\ obige Beispiele.  Im
Unterschied dazu tritt im sächsischen Sprachraum ein statistisch
signifikant erhöhter Naseninhalt auf, der sich in einer stärker
emotional geprägten Lautsprache niederschlägt.

\begin{figure}[tb]
  \centering
  \begin{tikzpicture}[thick,>=latex]
    \draw (0,0) -- (0,5);
    \draw[->] (0,0) -- (7,0);
    \node[below,anchor=north east] at (7,-0.1) {Anteil};
    \node[below] at (0,-0.1) {0\%};
    \draw (2.5,0.1) -- +(0,-0.2) node[below] {50\%};
    \draw (5,0.1) -- +(0,-0.2) node[below] {100\%};
    %
    \draw (0,1) rectangle node {S} (1,2);
    \draw (1,1) rectangle node {N} (3,2);
    \draw (3,1) rectangle node {D} (4,2);
    \draw (4,1) rectangle node {F} (5,2);
    \node[above] at (2.5,2) {Kontrollgruppe};
    %
    \draw (0,3) rectangle node {S} (1,4);
    \draw (1,3) rectangle node {N} (2,4);
    \draw (2,3) rectangle node {D} (4,4);
    \draw (4,3) rectangle node {F} (5,4);
    \node[above] at (2.5,4) {Testgruppe};
  \end{tikzpicture}
  \caption{Ergebnisse der Vergleichsuntersuchung des VHM, S =
    Sachinhalt, N = Naseninhalt, D = Dialektinhalt, F = Fremdinhalt.
    Dargestellt ist der Anteil der verschiedenen Ebenen von
    Nachrichten bei der Testgruppe und bei der Kontrollgruppe.}
  \label{fig:VHM}
\end{figure}

\section{Zusammenfassung}

Das Vier-Hälse-Modell stellt ein adäquates Modell zur Beschreibung
verschiedener Ebenen von Nachrichten im rheinischen Sprachraum dar.
Mit seiner Hilfe können subtile Botschaften, die mit Nachrichten
übermittelt werden, erkannt und entsprechend darauf reagiert werden
(z.B.\ durch Putzen, vgl.\ obige Beispiele).  Die Autoren sprechen
sich daher dafür aus, das Vier-Hälse-Modell in aktuelle Lehrbücher für
Philologie aufzunehmen und die im rheinischen Alltag übermittelten
Nachrichten auf die vier Ebenen hin zu untersuchen.

\section*{Danksagung}

Die Autoren danken Frau Irmgard Schorn und Herrn Paul Schorn für die
fruchtbaren Diskussionen und die Teilnahme an der
Vergleichsuntersuchung.

\bibliographystyle{ieeetr}
\bibliography{lit}

\end{document}
